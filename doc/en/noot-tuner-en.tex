\documentstyle[makeidx,texhelp]{report}
\newcommand{\noun}[1]{\textsc{#1}}

\makeindex

\begin{document}
\author{Davide Castellone}
\title{NOOT Instrument Tuner}
\date{August 2010}
\maketitle
\tableofcontents

\pagestyle{fancyplain}%

\chapter{Introduction}\label{introduction}

\noun{Never-out-of-tune Instrument Tuner} simulates a full-featured instrument tuner. It may
be used to tune musical instruments (guitars, pianos) or whatever
implies the detection of a sound frequency between 20 and 4000 Hz with a cent precision.

\noun{NOOT Instrument Tuner} is distributed under the
\helpref{GNU General Public License version 3}{license-gplv3}

Website: \urlref{http://code.google.com/p/noot-tuner/}{http://code.google.com/p/noot-tuner/}

Should you find any unexpected behaviour or want to make a suggestion or request a feature,
please go to the \urlref{bug reporting page}{http://code.google.com/p/noot-tuner/issues/entry}.

The author can be contacted on NOOT's mailing list:
\urlref{noot-tuner-discuss@googlegroups.com}{mailto:noot-tuner-discuss@googlegroups.com}

\section{Features}\label{features}

\begin{itemize}
	\item Easy to use (default settings will work in most cases)
	\item It works with almost any instrument
	\item Up to one-cent precision (0.01 semitones)
	\item Real-time display
	\item Octave and/or note selection
	\item Transposition
	\item Selectable temperament
	\item Internationalisation (English and Italian for now, other languages will follow)
	\item Cross-platform (it can be compiled on virtually any OS;
		Windows and Ubuntu packages available now,  other packages available soon)
	\item Basic documentation included
\end{itemize}

\textbf{Future plans}
\begin{itemize}
	\item Clock correction
	\item Ever more accuracy
	\item Customisable temperament
	\item More ports (OSX, ...) and languages
	\item PPA for Ubuntu users
	\item Input volume monitor, clipping detection and/or automatic volume
\end{itemize}

\section{Minimum requirements}\label{requirements}

\begin{itemize}
\item CPU clock frequency: at least 500 MHz
\item An audio card with a configured microphone
\item About 10 MB of RAM memory
\item About 7 MB of disk space
\end{itemize}

\section{Changelog}\label{changelog}

Version 0.1.0:

\itemize{\item First published version}

\chapter{Using \noun{NOOT Instrument Tuner}}\label{using}

Before starting, you need to configure your input device.
Usually this means choosing the input device and adjusting
its volume in the mixer.

After doing that, start \noun{NOOT}. Usually there is
no need for changing the default options.

Click on the \tt{Start/Stop} button to start the tuner.

If you already know the note you are tuning and do not need to
discover it, you should set the \helpref{note}{note} and
\helpref{octave}{octave} option.
This will make note detection easier.

\section{The main window}\label{mainwindow}

$$\image{}{pictures/mainwindow.gif}$$
\index{main window}
\caption{The main window}

Here is an explanation of all the controls in the main window.

\subsection{Start/Stop button}\label{startstop}

This button will start or stop the note detection. If another
application need to open the input device, you can temporarily
stop \noun{NOOT} and restart it later, without closing its
window.

\subsection{'Detected note' box}\label{detectednote}\index{note}

This box is made of two parts. The tuning indicator will show
how much the detected note is far from the expected note. Its
colour varies according to whether the pitch is higher (blue)
lower (red) or quite right (green). If the note is perfectly
tuned, the indicator should be under the black line.

The lower part of the box shows the following information:
\begin{itemize}
\item[Note] \index{note}The detected note. This value depends on whether you
have set the note and the octave in the control panel below. If
they are note set to \tt{Auto}, the note displayed will be exactly
the choosen one.
\item[Offset] \index{offset}The offset from the expected pitch, in cents (1 cent = 
0.01 semitones)
\item[Frequency] \index{frequency}The detected frequency in Hertz. This information is
useless to most of the users.
\end{itemize}

\subsection{Octave selector}\label{octave}\index{octave}

Selecting the right octaves often improves the precision of the
detection algorithm. When the octave is set to \tt{Auto}, the program
guesses it, sometimes leading to a little accurate measure.

With the standard used the middle C is in octave 3.

Sometimes the right octave is not the best, since every instrument
produces more than one harmonic. If the detection is not precise
enough (i.e. the indicator glitches too much) you can safely select
an octave lower or higher than the original one.
Remember that if you set the octave in the control panel, the detected
octave may not correspond to that of the note.

\subsection{Note selector}\label{note}\index{note}

Like for the octave, specifying the note oftens improves the precision,
but in most cases you should not need it.

\subsection{'Listen' button}\label{listen}

Click on this button to listen to the frequency of the note selected.

\subsection{Temperament selector}\label{temperament}

With \noun{NOOT Instrument Tuner} you can choose among a variety of temperaments.
The explanation of the temperaments listed is not a subject of this
manual.

\subsection{Transpose}\label{transpose}

Use this control to transpose the input. It can be a decimal number.

The frequency displayed above will not be altered.

\subsection{Window size}\label{windowsize}\index{FFT}

This option affects the note detection algorithm. It represents the number
of samples on which the FFT and autocorrelation are performed.

It should be set to a reasonable value (usually 4096).
If too low, the precision constraint may not be satisfied. If too high,
it may slow down the computer or even make the detection less precise.

\subsection{Threshold}\label{threshold}\index{noise}

This option is used to start the detection only if the input is loud enough,
in order to avoid glitches due to the background noise.

A threshold of 0 dB means that the detection will never start, while a threshold
of 90 dB (-90 dB, actually) will allow any input to be processed.

\subsection{Expected precision}\label{expectedprecision}\index{precision}

The lower is this number, the higher will be the precision. It represents the
theoretical error in cents on the detected frequency.

You can set this option to a higher value than 1 (the minimum value)
if you want to speed up the detection, at the cost of a slight loss of precision.

\subsection{Frame rate}\label{framerate}

This number represents the average number of updates per second. If it is set to 
a too high value, the program might hang.

\chapter{How it works}\label{howitworks}

\noun{NOOT Instrument tuner} uses a mix of FFT and linear autocorrelation to reach one-cent precision with relatively narrow window sizes.

Every time the display is updated, the following algorithm is computed on the current window:

\begin{itemize}
  \item A FFT is used to find the frequency with the highest spectral density within the note and octave constraints (if selected)
  \item From that frequency, an interval of possible frequencies is taken.
  \item Autocorrelation is computed at the times corresponding to that frequency interval; the time lag corresponding to the highest autocorrelation is kept.
  \item Repeatedly, the frequency is halved (the time lag is doubled) and the time lag corresponding to the highest autocorrelation among the near values is kept.
\end{itemize}

The main disadvantage of this approach is that it ceases to be precise in cases when the harmonics are far less than perfect.

\appendix

\chapter{License}\label{license}

\noun{NOOT Instrument Tuner} is distributed under the \helpref{GNU General Public License}{license-gplv3}

\include{gpl-3.0.tex}

\end{document}
