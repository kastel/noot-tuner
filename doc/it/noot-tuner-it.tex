\documentstyle[makeidx,texhelp]{report}
\newcommand{\noun}[1]{\textsc{#1}}

\makeindex

\begin{document}
\author{Davide Castellone}
\title{Accordatore NOOT}
\date{Agosto 2010}
\maketitle
\tableofcontents

\pagestyle{fancyplain}%

\chapter{Introduzione}\label{introduction}

L'accordatore \noun{NOOT} (\textit{Never-out-of-tune Instrument Tuner}) simula un
accordatore elettronico. Può essere usato per accordare uno strumento musicale
(chitarra, pianoforte) o qualsiasi cosa richieda il riconoscimento di
una frequenza sonora tra 20 e 4000 Hz con la precisione di un centesimo di semitono.

\copyright 2010 Davide Castellone

\noun{NOOT} è distribuito sotto la
\helpref{GNU General Public License version 3}{license-gplv3}

% Il logo e l'icona sono \copyright 2010 Carmen Buonfiglio, concessi sotto la licenza
%\urlref{Creative Commons Attribution-NonCommercial-NoDerivs 3.0 Unported License}{http://creativecommons.org/licenses/by-nc-nd/3.0/}

Sito web: \urlref{http://code.google.com/p/noot-tuner/}{http://code.google.com/p/noot-tuner/}

In caso di comportamento imprevisto del programma, o nel caso vogliate dare un suggerimento
o richiedere una funzionalità aggiuntiva, si prega di visitare la 
\urlref{pagina dei bug}{http://code.google.com/p/noot-tuner/issues/entry}.

L'autore può essere contattato attraverso la mailing list di NOOT: 
\urlref{noot-tuner-discuss@googlegroups.com}{mailto:noot-tuner-discuss@googlegroups.com}

\section{Caratteristiche}\label{features}

\begin{itemize}
	\item Facilità d'uso (le impostazioni predefinite funzionano nella maggior parte dei casi)
	\item Funziona con qualsiasi strumento
	\item Precisione di mezzo cent (0.005 semitoni)
	\item Display in tempo reale
	\item Ottave e note selezionabili
	\item Trasposizione
	\item Temperamento selezionabile
	\item Internazionalizzazione
	\item Cross-platform (può essere compilato praticamente in qualsiasi sistema operativo)
	\item Documentazione di base inclusa
\end{itemize}

\textbf{Sviluppi futuri}
\begin{itemize}
	\item Clock correction
	\item Ancora più precisione
	\item Temperamento completamente personalizzabile
	\item Più sistemi operativi e lingue
	\item Monitor del volume in ingresso, rilevazione del clipping e/o volume automatico
\end{itemize}

\section{Requisiti minimi}\label{requirements}

\begin{itemize}
%\item Fequenza di clock della CPU: almeno 500 MHz
\item Una scheda audio con un ingresso audio configurato
\item Circa 10MB di spazio su disco
\end{itemize}

\section{Changelog}\label{changelog}

Versione 0.1.1:
\itemize{
	\item La precisione è data in millesimi e non più in centesimi
	\item Cambiamenti minori al display
	\item PPA per gli utenti Ubuntu
}

Versione 0.1.0:

\itemize{\item Prima versione pubblicata}

\section{Riconoscimenti}\label{credits}

Codice e traduzione italiana \copyright 2010 Davide Castellone

\noun{NOOT} è distribuito con la licenza
\helpref{GNU General Public License version 3}{license-gplv3}

% Logo e icona \copyright 2010 Carmen Buonfiglio, sotto la licenza
%\urlref{Creative Commons Attribution-NonCommercial-NoDerivs 3.0 Unported License}{http://creativecommons.org/licenses/by-nc-nd/3.0/}

\chapter{Come si usa}\label{using}

Prima di iniziare, bisogna configurare il dispositivo di
ingresso. Solitamente basta selezionare il dispositivo
di ingresso nel mixer di sistema e regolare il volume.

Dopodiché, si può avviare \noun{NOOT}. Normalmente
non è necessario cambiare le impostazioni.

Cliccare sul pulsante \tt{Start/Stop} per avviare l'accordatore.

Se si conosce già la nota da accordare e non c'è quindi
necessità di rilevarla, si consiglia di impostare le
opzioni \helpref{nota}{note} e \helpref{ottava}{octave}.
Ciò renderà la misura più semplice per il programma.

\section{La finestra principale}\label{mainwindow}

$$\image{}{pictures/screenshot-it.png}$$
\index{main window}
\caption{Finestra principale}

Di seguito, la descrizione di tutti i controlli della finestra principale.

\subsection{Pulsante Start/Stop}\label{startstop}

Questo pulsante avvia o ferma la rilevazione. Se il dispositivo
di ingresso è richiesto da un'altra applicazione, si può
momentaneamente fermare \noun{NOOT} e riavviarlo successivamente,
senza chiudere la sua finestra.

\subsection{Riquadro "nota rilevata"}\label{detectednote}\index{note}

Questo riquadro si suddivide in due parti. L'indicatore di
intonazione mostra quanto la frequenza rilevata si allontani
dalla nota prevista. Se la nota è crescente, l'indicatore
è blu, se è calante l'indicatore è rosso; se è molto vicina
a quella corretta, l'indicatore è verde. Quando la nota
è perfettamente intonata, l'indicatore dovrebbe trovarsi
sotto la linea nera.

La parte inferiore del riquadro mostra le seguenti informazioni:

\begin{itemize}
\item[Nota rilevata] \index{note}La nota rilevata. Il suo valore cambia
se viene selezionata una nota o un'ottava nel pannello di controllo
sottostante. Se entrambe sono impostati su \tt{Auto}, la nota
mostrata sarà quella rilevata dal programma.
\item[Scostamento] \index{offset}Lo scostamento dalla frequenza prevista,
in cent (1 cent = 0.01 semitoni).
\item[Frequenza] \index{frequency}La frequenza rilevata in Hz. Questa informazione
è di poco uso.
\end{itemize}

\subsection{Ottava}\label{octave}\index{octave}

Selezionare l'ottava giusta spesso aumenta la precisione della misura.
Quando l'ottava è impostata su \tt{Auto}, il programma cerca di
indovinarla, portando spesso a misure meno precise.

Con lo standard adottato, il Do centrale si trova nell'ottava 3.

Talvolta l'ottava precisa non è la migliore, in quanto ogni strumento
produce più di un'armonica. Se la rilevazione non è abbastanza precisa
(cioè, se l'indicatore oscilla troppo) si può selezionare l'ottava
immediatamente superiore o inferiore a quella originale, senza rischi.
Si ricorda che se viene impostata un'ottava nel pannello di controllo,
l'ottava rilevata potrebbe non corrispondere a quella della nota.

\subsection{Nota}\label{note}\index{note}

Come per l'ottava, specificare la nota spesso migliora la precisione,
ma spesso non è necessario.

\subsection{Pulsante "ascolta"}\label{listen}

Cliccare su questo pulsante per ascoltare la nota selezionata.

\subsection{Selezione del temperamento}\label{temperament}

Con \noun{NOOT}, si può scegliere tra una varietà di temperamenti.
La descrizione dei temperamenti elencati non è argomento di questo
manuale.

\subsection{Trasponi}\label{transpose}

Usare questo controllo per trasporre l'input di un certo numero
di semitoni. Può essere un numero decimale. La frequenza
mostrata non sarà alterata.

The frequency displayed above will not be altered.

\subsection{Dimensione finestra}\label{windowsize}\index{FFT}

Questa opzione influenza l'algoritmo di rilevazione della frequenza.
Rappresenta il numero di campioni su cui vengono computate la FFT
e l'autocorrelazione.

Dovrebbe essere impostato su un valore ragionevole (per esempio 4096).
Se troppo basso, la precisione desiderata potrebbe non essere
raggiunta. Se troppo alto, potrebbe rallentare il computer o
persino peggiorare la misura.

\subsection{Soglia}\label{threshold}\index{noise}

Questa opzione serve ad avviare la rilevazione solo quando il livello
dell'input è abbastanza alto, così da evitare interferenze dovute al
rumore di sottofondo.

Una soglia di 0 dB significa che la rilevazione non si avvierà mai,
mentre una soglia di 90 dB (-90 in realtà) lascerà passare
qualsiasi input.

A threshold of 0 dB means that the detection will never start, while a threshold
of 90 dB (-90 dB, actually) will allow any input to be processed.

\subsection{Precisione prevista}\label{expectedprecision}\index{precision}

Minore il valore, maggiore la precisione. Rappresenta l'errore
teorico in millesimi di semitono sulla frequenza misurata.

Valori più bassi rendono il programma meno reattivo e
possono rallentarlo fino a farlo bloccare.

\subsection{Frame rate}\label{framerate}

Questo numero rappresenta il numero medio di aggiornamenti al
seconto. Se impostato su un valore troppo alto, il programma
potrebbe bloccarsi.

\chapter{Come funziona}\label{howitworks}

L'accordatore \noun{NOOT} usa un misto di FFT e autocorrelazione
lineare per raggiungere la precisione di un cent con
finestre di larghezza relativamente ridotta.

Ogni volta che il display è aggiornato, viene eseguito il seguente algoritmo:
\begin{itemize}
  \item La FFT viene usata per trovare la frequenza con la massima densità spettrale entro la nota e l'ottava selezionate.
  \item Da quella frequenza, viene ricavato un intervallo di frequenze possibili.
  \item Si calcola l'autocorrelazione ai tempi corrispondenti a quell'intervallo di frequenze; viene tenuto il tempo che corrisponde all'autocorrelazione più alta.
  \item Ripetutamente, la frequenza è dimezzata (il tempo è raddoppiato) e viene tenuto il tempo corrispondente all'autocorrelazione più alta tra i valori vicini.
\end{itemize}

Lo svantaggio principale di questo approccio è la perdita di precisione nei casi in cui gli armonici sono lontani dalla perfezione.

\appendix

\chapter{Licenza}\label{license}

\noun{NOOT Instrument Tuner} è distribuito sotto la \helpref{GNU General Public License}{license-gplv3}

\include{gpl-3.0.tex}

\end{document}
